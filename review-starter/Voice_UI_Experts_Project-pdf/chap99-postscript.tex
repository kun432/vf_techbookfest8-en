\chapter{あとがき}
\label{chap:chap99-postscript}

いかがだったでしょうか。
感想や質問は随時受けつけています。
\\{}
\\{}

\subsection*{著者紹介}
\addcontentsline{toc}{subsection}{著者紹介}
\label{sec:-0-1}
{
  \def\starterminiimageframe{Y}
  \begin{startersideimage}{L}{./images/chap99-postscript/tw-icon.jpg}{20truemm}{20truemm}{7truemm}{}
\noindent
\reviewstrong{カウプラン機関極東支部}\\{}
\href{https://twitter.com/_kauplan/}{@\textunderscore{}kauplan}\\{}
\href{https://kauplan.org/}{https://kauplan.org/}\\{}
「\href{https://www.amazon.co.jp/dp/4063348792}{パンプキン・シザーズ}」は傑作(カウプラン特許すごすぎぃぃ!)。\\{}
「\href{http://worldtrigger.info/}{ワールド・トリガー}」は傑作(祝連載再開!たしかなまんぞく)。\\{}
「\href{https://pripri-anime.jp/}{プリンセス・プリンシパル}」は傑作(5話のアクションだいすき)。\\{}

  \end{startersideimage}
}
\vspace*{\baselineskip}

\subsection*{既刊一覧}
\addcontentsline{toc}{subsection}{既刊一覧}
\label{sec:-0-2}

\begin{starteritemize}
\item 『SQL高速化 in PostgreSQL』(技術書典2)
\item 『オブジェクト指向言語解体新書』(技術書典3)
\item 『\href{https://kauplan.org/books/jquery/}{jQueryだって複雑なアプリ作れるもん!}』(技術書典4)
\item 『\href{https://kauplan.org/books/serversetup/}{Shellスクリプトでサーバ設定を自動化する本}』(技術書典5)
\item 『\href{https://kauplan.org/books/errmsg/}{Rubyのエラーメッセージが読み解けるようになる本}』(技術書典6)
\item 『\href{https://kauplan.org/books/sql/}{わかりみSQL}』(技術書典7)
\end{starteritemize}
